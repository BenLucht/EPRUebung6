\documentclass[fleqn]{article}
\usepackage[left=1in, right=1in, top=1in, bottom=1in]{geometry}
\usepackage{mathexam}
\usepackage{amsmath}
\usepackage{hhline}
\usepackage{float}
\usepackage{paralist}
\usepackage{enumerate}
\usepackage{pdfpages}
\newcommand{\shellcmd}[1]{\texttt{\footnotesize\ #1}\\}
\usepackage[utf8]{inputenc}

\ExamName{xxxxxxx: Ben, xxxxxxx: Anne}
\ExamClass{EPR Übung 6\\}
\ExamHead{29. Dezember 2017}

\let\ds\displaystyle

\def\nand{\bar\land}
\def\nor{\bar\lor}

\begin{document}

\begin{enumerate}[{Aufgabe} 6.2]
\item
    Sowohl Pickle als auch JSON sind Arten der Serialisierung von Daten. Pickle
    ist dabei das Format der Speicherung von Objekten in Binärform, die in Python
    vorgesehen ist. Wenn das Objekt nur zur Verwendung in Python-Programmen
    (zwischen-)gespeichert werden soll, dann ist Pickle als Format ausreichend.
    Soll die gespeicherte Datei auch von anderen Programmiersprachen verwendet werden,
    ist das JSON-Format sinnvoller, da die Verarbeitung des Standards in den meisten
    Sprachen implementiert ist oder entsprechende Module leicht zu finden sind.
    Im Gegensatz zu Pickle-Dateien ist das JSON-Format textbasiert und somit für
    Menschen lesbar. In Pickle-Dateien kann schadhafter Programmcode enthalten sein,
    der beim laden der Datei ausgeführt wird.
\end{enumerate}

\begin{enumerate}[{Aufgabe} 6.4]
\item
    Text.
\end{enumerate}

\end{document}